%% The default is a single spaced, 10 point font, single spaced article.
%% There are 5 other style options available via an optional argument. They
%% can be envoked like this:
%%
%% \documentclass[argument]{aastex62}
%% 
%% where the layout options are:
%%
%%  twocolumn   : two text columns, 10 point font, single spaced article.
%%                This is the most compact and represent the final published
%%                derived PDF copy of the accepted manuscript from the publisher
%%  manuscript  : one text column, 12 point font, double spaced article.
%%  preprint    : one text column, 12 point font, single spaced article.  
%%  preprint2   : two text columns, 12 point font, single spaced article.
%%  modern      : a stylish, single text column, 12 point font, article with
%% 		  wider left and right margins. This uses the Daniel
%% 		  Foreman-Mackey and David Hogg design.
%%  RNAAS       : Preferred style for Research Notes which are by design 
%%                lacking an abstract and brief. DO NOT use \begin{abstract}
%%                and \end{abstract} with this style.
%%
%% Note that you can submit to the AAS Journals in any of these 6 styles.
%%
%% There are other optional arguments one can envoke to allow other stylistic
%% actions. The available options are:
%%
%%  astrosymb    : Loads Astrosymb font and define \astrocommands. 
%%  tighten      : Makes baselineskip slightly smaller, only works with 
%%                 the twocolumn substyle.
%%  times        : uses times font instead of the default
%%  linenumbers  : turn on lineno package.
%%  trackchanges : required to see the revision mark up and print its output
%%  longauthor   : Do not use the more compressed footnote style (default) for 
%%                 the author/collaboration/affiliations. Instead print all
%%                 affiliation information after each name. Creates a much
%%                 long author list but may be desirable for short author papers
%%
%% these can be used in any combination, e.g.
%%
%% \documentclass[twocolumn,linenumbers,trackchanges]{aastex62}
%%
%% AASTeX v6.* now includes \hyperref support. While we have built in specific
%% defaults into the classfile you can manually override them with the
%% \hypersetup command. For example,
%%
%%\hypersetup{linkcolor=red,citecolor=green,filecolor=cyan,urlcolor=magenta}
%%
%% will change the color of the internal links to red, the links to the
%% bibliography to green, the file links to cyan, and the external links to
%% magenta. Additional information on \hyperref options can be found here:
%% https://www.tug.org/applications/hyperref/manual.html#x1-40003
 \documentclass[twocolumn,floatfix,tighten]{aastex62}
%\documentclass[twocolumn,floatfix,tighten]{aastex}

\usepackage{amsmath}
\usepackage{amssymb}
\usepackage{graphicx}
\usepackage{bm}

\bibliographystyle{aasjournal}

% Units
%\input{src/shorthands}

% Some fancy commenting
\definecolor{todo}{RGB}{200,0,0}
\newcommand{\note}[2][todo]{{\color{#1}[[{\bf #2}]]}}

\received{}
\revised{}
\accepted{}
\submitjournal{AJ}

%\AuthorCollaborationLimit=3

%% Please do not use
%% this functionality for manuscripts with less than 20 authors. Conversely,
%% please do use this when the number of authors exceeds 40.
%% Use \allauthors at the manuscript end to show the full author list.
%% This command should only be used with \AuthorCollaborationLimit is used.

\shortauthors{DiGiorgio et al.}
\shorttitle{Bisymmetric Modes in SDSS-IV/MaNGA}

\begin{document}

\title{The Strength of Bisymmetric Modes in SDSS-IV/MaNGA Kinematics}

%% A significant change from earlier AASTEX versions is in the structure for 
%% calling author and affilations. The change was necessary to implement 
%% autoindexing of affilations which prior was a manual process that could 
%% easily be tedious in large author manuscripts.
%%
%% The \author command is the same as before except it now takes an optional
%% arguement which is the 16 digit ORCID. The syntax is:
%% \author[xxxx-xxxx-xxxx-xxxx]{Author Name}
%%
%% This will hyperlink the author name to the author's ORCID page. Note that
%% during compilation, LaTeX will do some limited checking of the format of
%% the ID to make sure it is valid.
%%
%% Use \affiliation for affiliation information. The old \affil is now aliased
%% to \affiliation. AASTeX v6.2 will automatically index these in the header.
%% When a duplicate is found its index will be the same as its previous entry.
%%
%% Note that \altaffilmark and \altaffiltext have been removed and thus 
%% can not be used to document secondary affiliations. If they are used latex
%% will issue a specific error message and quit. Please use multiple 
%% \affiliation calls for to document more than one affiliation.
%%
%% The new \altaffiliation can be used to indicate some secondary information
%% such as fellowships. This command produces a non-numeric footnote that is
%% set away from the numeric \affiliation footnotes.  NOTE that if an
%% \altaffiliation command is used it must come BEFORE the \affiliation call,
%% right after the \author command, in order to place the footnotes in
%% the proper location.
%%
%% Use \email to set provide email addresses. Each \email will appear on its
%% own line so you can put multiple email address in one \email call. A new
%% \correspondingauthor command is available in V6.2 to identify the
%% corresponding author of the manuscript. It is the author's responsibility
%% to make sure this name is also in the author list.
%%
%% While authors can be grouped inside the same \author and \affiliation
%% commands it is better to have a single author for each. This allows for
%% one to exploit all the new benefits and should make book-keeping easier.
%%
%% If done correctly the peer review system will be able to
%% automatically put the author and affiliation information from the manuscript
%% and save the corresponding author the trouble of entering it by hand.

\correspondingauthor{Brian DiGiorgio}
\email{bdigiorg@ucsc.edu}

\author{Brian DiGiorgio}
\affiliation{Department of Astronomy and Astrophysics, University of California, Santa Cruz, 1156 High St., Santa Cruz, CA 95064, USA}

\author[0000-0003-1809-6920]{Kyle B. Westfall}
\affiliation{University of California Observatories, University of California, Santa Cruz, 1156 High St., Santa Cruz, CA 95064, USA}

\author{Niv Drory}
\affiliation{ McDonald Observatory, University of Texas at Austin, 1 University Station, Austin, TX 78712, USA}

\author{Matthew A. Bershady}
\affiliation{Department of Astronomy, University of Wisconsin-Madison, 475N. Charter St., Madison WI 53703, USA}

\author{Kevin Bundy}
\affiliation{Department of Astronomy and Astrophysics, University of California, 1156 High Street, Santa Cruz, CA 95064, USA}
\affiliation{University of California Observatories, University of California, 1156 High Street, Santa Cruz, CA 95064, USA}

\author{Stephanie Campbell}
\affiliation{School of Physics and Astronomy, University of St. Andrews, North Haugh, St. Andrews KY16 9SS, UK}

\author{Anne-Marie Weijmans}
\affiliation{School of Physics and Astronomy, University of St. Andrews, North Haugh, St. Andrews KY16 9SS, UK}

\author{Karen Masters}
\affil{Department of Physics and Astronomy, Haverford College, 370 Lancaster Avenue, Haverford, PA 19041, USA}

\author{David Stark}
\affil{Department of Physics and Astronomy, Haverford College, 370 Lancaster Avenue, Haverford, PA 19041, USA}

\author{David Law}
\affiliation{Space Telescope Science Institute, 3700 San Martin Drive, Baltimore, MD 21218, USA}

\begin{abstract}

Representing the largest sample of kinematic maps obtained to date, the
SDSS-IV/MaNGA Survey data provide an unprecendented opportunity to study
the internal motions of galaxies.  We present forward modeling results
for a nonaxisymmetric kinematic model that decomposes MaNGA velocity
fields into the radial and tangential modes up to second order for both
ionized gas and stars. This approach allows us to quantify the influence
of noncircular, bisymmetric features like bars and warps, as well as
producing kinematic geometries, rotation curves, and velocity-dispersion
profiles.  We present our modeling approach, tests of its efficacy, and
correlate the strength of the non-axisymmetric kinematic components with
galaxy morphology and spectral energy distributions.  Both our analysis
software and the resulting data are made publicly available.

\end{abstract}

%\startlongtable
%\begin{deluxetable}{c|cc}
%\tablecaption{ApJ costs from 1991 to 2013\tablenotemark{a} \label{tab:table}}
%\tablehead{
%\colhead{Year} & \colhead{Subscription} & \colhead{Publication} \\
%\colhead{} & \colhead{cost} & \colhead{charges\tablenotemark{b}}\\
%\colhead{} & \colhead{(\$)} & \colhead{(\$/page)}
%}
%\colnumbers
%\startdata
%1991 & 600 & 100 \\
%...
%\enddata
%\tablenotetext{a}{Adjusted for inflation}
%\tablenotetext{b}{Accounts for the change from page charges to digital quanta in April, 2011}
%\tablecomments{Note that {\tt \string \colnumbers} does not work with the 
%vertical line alignment token. If you want vertical lines in the headers you
%can not use this command at this time.}
%\end{deluxetable}

% In AASTeX v6.2 all deluxetables are float tables and thus if they are
% longer than a page will spill off the bottom. Long deluxetables should
% begin with the {\tt\string\startlongtable} command. This initiates a
% longtable environment.  Authors might have to use {\tt\string\clearpage} to
% isolate a long table or optimally place it within the surrounding text.

\section{Introduction}
\label{sec:intro}

\section{MaNGA Data}
\label{sec:data}


\subsection{Data Processing} \label{sec:clipping}

We must first sanitize the velocity data from each galaxy to make it more suitable for model fitting. Though the MaNGA DAP masks many imperfections in the maps it extracts from the data cubes, there are still outliers in the data that inhibit our ability to produce a successful fit. 

First, we mask any bins that consist of more than 10 spaxels. Such bins appear frequently on the outskirts of stellar velocity fields where the MaNGA DAP groups spaxels together to reach a sufficient S/N. However, such large bins frequently cause problems because it is difficult to define their centers properly when fitting a velocity field and models with large bins do not behave well when smeared with a PSF. 

To remove spurious velocity measurements that are not tied to any real physical motion within the galaxy, we convolve a PSF over the velocity field and velocity dispersion. Doing so will average out any incorrectly extracted velocity/dispersion measurements over an observationally realistic region. We then mask any spaxels where the magnitude of the difference between the velocity and dispersion maps and their doubly-smeared counterparts is more than 50 km/s since deviations that large are likely in error.

We then mask out any spaxels that have a surface brightness flux of less than $3 \times 10^{-19}$ ergs/s/cm$^2$ or an amplitude-to-noise ratio (ANR) of less than 5. These values were visually determined to best remove low-quality velocity measurements on the outskirts of galaxies that tend to have underreported errors from the DAP.

Finally, we attempt to remove any regions of the velocity field that do not appear to be part of the same rotating system as the rest of the galaxy. Many MaNGA IFUs contain foreground/background sources or merging companions that have distinct velocity fields from the main target, so it would be inappropriate to fit a single rotating disk to the data. To mask these, we iteratively fit the velocity field with our simple axisymmetric model described in Section \ref{sec:axisym} and subtract the model from the data to obtain a map of the residuals. For an orderly axisymmetric velocity field well-described by this model, these residuals should be Gaussian, and any deviations from Gaussianity represent possible signatures of axisymmetry that we may want to mask. In order to preserve the genuine bisymmetric features we are attempting to model, we mask only the spaxels that differ from the mean of the residuals by more than 10 standard deviations, a value we determined removes unwanted companions but still preserves real bisymmetric features. After masking these spaxels, we again fit the axisymmetric model and remove the outliers in the residuals, repeating the process until the number of masked spaxels stabilizes.

If, at the end of this process, the galaxy is left with only 5\% or less of its original number of spaxels unmasked, the velocity field is considered to be unsuitable for velocity field fitting and it is not fit. Of the full MaNGA sample, approximately 25\% of velocity fields fall into this category, so the fitting algorithm is only run on XXXX out of XXXX gas velocity fields and XXXX out of XXXX stellar velocity fields \note{update with actual numbers}.

\section{Forward Modeling Approach}
\label{sec:modeling}

\subsection{Axisymmetric Kinematic Model} \label{sec:axisym}

\note{Kyle}

\subsection{Bisymmetric Kinematic Model} \label{sec:bisym}
To model noncircular motions in disk galaxies, we adopt a formalism based on \cite{spekkens07}. In a thin rotating disk with no out-of-plane motion, all motion may be broken down into a tangential component $V_t$ and a radial component $V_r$, or

\begin{equation} \label{components}
    V(r,\theta) = V_t(r,\theta) + V_r(r,\theta)
\end{equation}

\noindent for deprojected polar coordinates $r$ and $\theta$. These components can then be expressed as a Fourier series within the plane of the disk:

\begin{equation}
    V_t(r, \theta) = V_{0,t}(r) + \sum_{m=1}^\infty V_{m,t}(r) \cos \bigg( m \big(\theta + \phi_{m,t}(r) \big) \bigg)
\end{equation}

and

\begin{equation}
    V_r(r, \theta) = V_{0,r}(r) + \sum_{m=1}^\infty V_{m,r}(r) \cos \bigg( m \big(\theta + \phi_{m,r}(r) \big) \bigg),
\end{equation}

\noindent where the coefficients $V_{m,t}$ and $V_{m,r}$ represent the relative amplitudes of each velocity mode in either component and $\phi_{m,t}$ and $\phi_{m,r}$ represent the relative phase angles of each component in a given annulus. 

However, galaxies are observed projected onto the sky plane, rotated at an inclination angle $i$ about their major axis and moving at some line-of-sight systemic velocity $V_{sys}$ towards or away from the observer. We will assume $\theta=0$ to be the major axis, the azimuth where the line-of-sight velocity is greatest away from the observer. With these adjustments, Equation \ref{components} can be adapted to express the line-of-sight velocity $V_{obs}$:

\begin{equation}
    V_{obs} = V_{sys} + \sin i \, \big(V_t(r,\theta) + V_r(r,\theta) \big).
\end{equation}

The above expression is exactly correct if we use infinite terms in the Fourier series and have infinite radial resolution in the velocity coefficients. However, when fitting a model to data, we cannot realistically compute infinitely many Fourier coefficients, so for the purposes of this paper, we will restrict ourselves only to the first- and second-order tangential components and the second-order radial components. We justify capping our analysis at the second order by noting that bars, warps, and spiral arms, are bisymmetric, meaning that modes higher than $m=2$ will mostly only fit local kinematic perturbations rather than larger dynamical features.\note{cite} We do not include a first-order radial term because a large coefficient there would violate the continuity theorem and we do not expect most galaxies to have large inflows or outflows.

Within the first two orders, we make some additional simplifications. First, we restrict that the second-order components share a position angle $\phi_b$ relative to the first-order position angle, meaning that we assume the bisymmetric distortion is a single physical disturbance with both radial and tangential components. We also adopt the phase conventions of \cite{spekkens07}, imposing a $\pi/2$ phase shift to the radial component. With these simplifications, the model we adopt for the remainder of this paper is the following:

\begin{multline} \label{model}
    V(r, \theta) = V_{sys} + \sin i \, \big(V_t(r) \cos \theta - V_{2t}(r) \cos2 (\theta - \phi_b) \cos \theta \\ - V_r(r) \sin 2 (\theta - \phi_b) \sin \theta \big).
\end{multline}

When working with real on-sky data, we must first map the rectilinear on-sky coordinates of the spectroscopic observations to the deprojected polar coordinates we use in the model using the following transformations:

\begin{equation}
    r = \sqrt((x - x_c)^2 + (y - y_c)^2)
\end{equation}
\begin{equation}
    \theta = \arctan\left(\frac{x \sin \phi - y \cos \phi}{\cos i\, (x \cos \phi + y \sin \phi)}\right),
\end{equation}

\noindent for $x$ and $y$ center position $x_c$ and $y_c$ and first-order position angle $\phi$. 


\subsection{Fitting Algorithm}

To fit the above model to the data appropriately, we construct a Bayesian forward model. We choose this formalism rather than a least-squares optimizer like \citep{spekkens07} because of its ability to accurately characterize multimodal posteriors and covariances between parameters.


We then construct the prior based off of the information we already have about the galaxy. Using the results from an initial fit using our least-squares axisymmetric fitting algorithm \note{Kyle needs to describe this}, we

using \texttt{dynesty} \citep{dynesty}, a Python package implementing nested sampling \citep{skilling04, skilling06} utilizing multi-ellipsoid bounds \citep{feroz09}. The algorithm for fitting is as follows.



\subsection{Fit Assessments}

\section{Results}

\section{Summary}

\software{Astropy \citep{2013A&A...558A..33A, 2018AJ....156..123A};
Numpy \citep{harris2020array};
Scipy \citep{2020SciPy-NMeth}; matplotlib \citep{Hunter:2007}}

\note{Add fftw}

\acknowledgements

We acknowledge ...

We acknowledge use of the lux supercomputer at UC Santa Cruz, funded by NSF MRI grant AST 1828315.

Funding for the Sloan Digital Sky Survey IV has been provided by the
Alfred P. Sloan Foundation, the U.S. Department of Energy Office of
Science, and the Participating Institutions. SDSS-IV acknowledges
support and resources from the Center for High-Performance Computing at
the University of Utah. The SDSS web site is www.sdss.org.

SDSS-IV is managed by the Astrophysical Research Consortium for the
Participating Institutions of the SDSS Collaboration including the
Brazilian Participation Group, the Carnegie Institution for Science,
Carnegie Mellon University, the Chilean Participation Group, the French
Participation Group, Harvard-Smithsonian Center for Astrophysics,
Instituto de Astrof\'isica de Canarias, The Johns Hopkins University,
Kavli Institute for the Physics and Mathematics of the Universe (IPMU) /
University of Tokyo, Lawrence Berkeley National Laboratory, Leibniz
Institut f\"ur Astrophysik Potsdam (AIP),  Max-Planck-Institut f\"ur
Astronomie (MPIA Heidelberg), Max-Planck-Institut f\"ur Astrophysik (MPA
Garching), Max-Planck-Institut f\"ur Extraterrestrische Physik (MPE),
National Astronomical Observatories of China, New Mexico State
University, New York University, University of Notre Dame,
Observat\'ario Nacional / MCTI, The Ohio State University, Pennsylvania
State University, Shanghai Astronomical Observatory, United Kingdom
Participation Group, Universidad Nacional Aut\'onoma de M\'exico,
University of Arizona, University of Colorado Boulder, University of
Oxford, University of Portsmouth, University of Utah, University of
Virginia, University of Washington, University of Wisconsin, Vanderbilt
University, and Yale University.

%\appendix
%
%\input{src/datamodel}
%\onecolumngrid

%% To help institutions obtain information on the effectiveness of their 
%% telescopes the AAS Journals has created a group of keywords for telescope 
%% facilities.
%
%% Following the acknowledgments section, use the following syntax and the
%% \facility{} or \facilities{} macros to list the keywords of facilities used 
%% in the research for the paper.  Each keyword is check against the master 
%% list during copy editing.  Individual instruments can be provided in 
%% parentheses, after the keyword, but they are not verified.
%
%\vspace{5mm}
%\facilities{HST(STIS), Swift(XRT and UVOT), AAVSO, CTIO:1.3m,
%CTIO:1.5m,CXO}
%
%% Similar to \facility{}, there is the optional \software command to allow 
%% authors a place to specify which programs were used during the creation of 
%% the manusscript. Authors should list each code and include either a
%% citation or url to the code inside ()s when available.
%
%\software{astropy \citep{2013A&A...558A..33A},  
%          Cloudy \citep{2013RMxAA..49..137F}, 
%          SExtractor \citep{1996A&AS..117..393B}
%          }

\bibliography{ref}

\end{document}

